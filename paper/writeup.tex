\documentclass[10pt]{article}
\usepackage[utf8]{inputenc}
\usepackage[english]{babel}
% \usepackage[margin=1in]{geometry}
% \newcommand\hmmax{0}
% \newcommand\bmmax{0}

% % % Fonts% %
   \usepackage[T1]{fontenc}
   % \usepackage{textcomp}
   % \usepackage{newtxtext}
   % \renewcommand\rmdefault{Pym} %\usepackage{mathptmx} %\usepackage{times}
   \usepackage[complete, subscriptcorrection, slantedGreek, mtpfrak, mtpbbi, mtpcal]{mtpro2}
   \usepackage{bm}% Access to bold math symbols
   % \usepackage[onlytext]{MinionPro}
   \usepackage[no-math]{fontspec}
   \defaultfontfeatures{Ligatures=TeX,Numbers={Proportional}}
   \newfontfeature{Microtype}{protrusion=default;expansion=default;}
   \setmainfont[Ligatures=TeX]{TimesNRMTPro}
   \setsansfont[Microtype,Scale=MatchLowercase,Ligatures=TeX,BoldFont={* Semibold}]{Myriad Pro}
   \setmonofont[Scale=0.8]{Atlas Typewriter}

   % \usepackage{selnolig}% For suppressing certain typographic ligatures automatically
   \usepackage{microtype}
% % % % % % % % %
\usepackage{amsthm}         % (in part) For the defined environments
\usepackage{mathtools}      % Improves  on amsmaths/mtpro2
\usepackage{bbding}         % For hand pointers, etc.

% % The bibliography % % %
\usepackage[backend=biber,
            style=authoryear-comp,
            citestyle=authoryear-comp,
            backref=false,
            hyperref=true,
            url=false,
            isbn=false,
           ]{biblatex}
% % \renewcommand*{\bibfont}{\small}
\DeclareFieldFormat{postnote}{#1}
\DeclareFieldFormat{multipostnote}{#1}
% \setlength\bibitemsep{1.5\itemsep}
\addbibresource{ling245.bib}

% \newcommand{\seccite}[1]{\citeauthor{#1}, \citetitle{#1}, \citeyear{#1}}
% % % % % % % % % % % % % % %

% % % The following section relates to theorems, etc. % % %
\usepackage{thmtools}
% \usepackage{thm-restate} % For restating theorems

\declaretheoremstyle[
spaceabove=6pt, spacebelow=6pt,
headfont=\normalfont\bfseries,
notefont=\mdseries, notebraces={(}{)},
bodyfont=\normalfont,
% postheadspace=1em,
% qed=\qedsymbol
]{defstyle}

\declaretheoremstyle[
spaceabove=6pt, spacebelow=6pt,
headfont=\normalfont\bfseries,
notefont=\normalfont\bfseries, notebraces={}{},
bodyfont=\normalfont,
% postheadspace=1em,
% qed=\qedsymbol
]{defsstyle}

\declaretheorem[name=Theorem,numberwithin=section]{theorem}
\declaretheorem[sibling=theorem,style=remark]{remark}
\declaretheorem[sibling=theorem,name=Corollary]{corollary}
\declaretheorem[sibling=theorem,name=Lemma]{lemma}
\declaretheorem[sibling=theorem,name=Fact]{fact}
\declaretheorem[sibling=theorem,name=Proposition]{proposition}
\declaretheorem[sibling=theorem,name=Argument]{argument}
\declaretheorem[sibling=theorem,name=Definition,style=defstyle]{definition}
\declaretheorem[name=Definitions,numbered=no,style=defsstyle]{definitions}
\declaretheorem[sibling=theorem,name=Example,style=defstyle]{example}
\declaretheorem[sibling=theorem,name=Scenario,style=defstyle]{scenario}

% % % % % % % % % % % % % % % % % % % % % % % % % % % % % %

% % % Custom Commands % % %
% \newcommand{\subfor}[2]{[\sfrac{#1}{#2}]}
\usepackage{xfrac,nicefrac} % For \sfrac{i}{j} and for nicer fractions
\newcommand{\sem}[1]{\ensuremath{[\kern-.5mm[{#1}]\kern-.5mm]}}
% % % % % % % % % % % % % %

\usepackage[inline]{enumitem}
        % \setlist[enumerate]{itemsep=.03125em}
        \setlist[itemize]{noitemsep}
        \setlist[description]{noitemsep,style=unboxed,leftmargin=.5cm,font=\normalfont\space}
        \setlist[enumerate]{noitemsep}

% % For Figures % % %
\usepackage{tikz,subfig} % For drawings

    \tikzstyle{index on}=[inner sep=2pt, white, circle, fill=black]
    \tikzstyle{index off}=[inner sep=2pt, black, circle, draw]
    \tikzstyle{index gray}=[inner sep=2pt, black, circle, fill=lightgray]
    \tikzstyle{opaque}=[fill=gray,fill opacity=.1]
    \tikzstyle{counter}=[densely dashed]
\usepackage{wrapfig}
\usepackage{float} % For correctly placed floats
% % % % % % % % % % %
% \usepackage{graphicx}

% % % Misc packages % % %

% \usepackage{refcheck} % Can be used for checking references
% \usepackage{lineno}   % For line numbers
\usepackage{multicol} % For multiple columns
% \usepackage{mathrsfs} % For elegant Latin math letters
\usepackage{proof}    % For natural deduction proofs
% \usepackage{hyphenat} % For \hyp{} hyphenation command, and general hyphenation stuff
% \usepackage{titling} % for multiple titles
% % % % % % % % % % % % %

\usepackage{chngcntr}
\counterwithout{paragraph}{subsubsection}
\renewcommand{\theparagraph}{\arabic{paragraph}.}
\setcounter{secnumdepth}{4}
\usepackage{titlesec}
% \titleformat{\paragraph}[runin]{\normalfont}{\indent}{\wordsep}{}
% \titlespacing{\paragraph}{0pt}{3.25ex plus 1ex minus .2ex}{5\wordsep}
% \titleformat{\paragraph}[runin]{\normalfont}{}{\wordsep}{}
% \titlespacing{\paragraph}{0pt}{1ex plus .2ex minus .2ex}{0pt}
\titleformat{\paragraph}[runin]{\normalfont}{\indent\theparagraph}{\wordsep}{}
\titlespacing{\paragraph}{0pt}{3.25ex plus 1ex minus .2ex}{5\wordsep}

\usepackage{adjustbox}

\usepackage[hidelinks,breaklinks]{hyperref}



\title{Ling 245 Class Project Writeup}
\author{Benjamin Sparkes}
% \date{}

\begin{document}

\maketitle

\section{Introduction}
\label{sec:introduction}

Recreating Experiment 1 from \textcite{Bott:2016aa}.

\section{Things to Note}
\label{sec:things-note}

Need a decent number of observations from each individual on each trial for the analysis \citeauthor{Bott:2016aa} perform to work.
Here, `decent number' means that we need at least one primeStrength and (WithCat/BetCat) trial with correct prime choices, else there are going to be fewer observations than random effects.

\begin{table}[h]
  \caption{Experiment 1 results.}\vspace{-20pt}
  \begin{center}
    \begin{adjustbox}{width=1.2\textwidth,center=\textwidth}
    \begin{tabular}{llrrrr}
      \hline
      & & \(\beta\) & S.E.\ & \emph{Z} & \emph{p}-value  \\
      \hline
      Overview & \multicolumn{2}{l}{\(\text{Prime} * \text{WithBet} + (1 + \text{Prime} * \text{WithBet} \mid \text{subject})\)}  & & & \\
      & (Intercept) & -0.594 & 0.198 & -2.991 & .003 \\
      & Prime & 0.563 & 0.034 & 16.342 & <.001 \\
      & WithBet & 0.126 & 0.029 & 4.284 & <.001 \\
      & Prime:WithBet & -0.430 & 0.033 & -13.177 & <.001 \\
      Within simple & Prime & 0.993 & 0.059 & 16.950 & <.001 \\
      Between Simple & Prime & 0.133 & 0.033 & 4.082 & <.001 \\
      Within detail & \multicolumn{2}{l}{\(\text{Prime} * \text{WithCat} + (1 + \text{Prime} * \text{WithCat} \mid \text{subject})\)}  & & & \\
      & (Intercept)  & -2.088 & 0.255 & -8.185 & <.001\\
      & Prime & 1.239 & 0.109 & 11.374 & <.001 \\
      & WithCatNUM4 & 2.068 & 0.195 & 10.588 & <.001 \\
      & WithCatSOME & 1.823 & 0.157 & 11.598 & <.001 \\
      & Prime:WithCatNUM4 & 0.174 & 0.166 & 1.046 & .269 \\
      & Prime:WithCatSOME & -0.138 & 0.137 & -1.007 & .314 \\
      Between detail & \multicolumn{2}{l}{\(\text{Prime} * \text{BetCat} + (1 + \text{Prime} * \text{BetCat} \mid \text{subject})\)}  & & & \\
      & (Intercept)  & -0.691 & 0.204 & -3.384 & <.001\\
      & Prime & 0.145 & 0.058 & 0.058 & .012 \\
      & BetCatSOMEADH & -0.054 & 0.089 & -0.611 & .540 \\
      & BetCatSOMENUM4 & 0.889 & 0.112 & 7.915 & <.001 \\
      & Prime:BetCatSOMEADH & -0.069 & 0.079 & -0.873 & .383 \\
      & Prime:BetCatSOMENUM4 & 0.078 & 0.088 & 0.888 & .374 \\
      \hline
    \end{tabular}
  \end{adjustbox}
  \end{center}

% \begin{adjustbox}{width=1.2\textwidth,center=\textwidth}
  % {\footnotesize
    \emph{Note}. R-pseudo code shown in the first line of every section.
  \emph{Prime} = priming factor (2 levels: strong, weak).
  \emph{WithBet} = within/between factor (2 levels: within, between).
  \emph{WithCat} = within expression category factor (3 levels: \emph{some}, number4, \emph{ad hoc}).
  \emph{Betcat} = between expression category factor (3 levels: \emph{some} \(\leftrightarrow\) number4, \emph{some} \(\leftrightarrow\) \emph{ad hoc}, number4 \(\leftrightarrow\) \emph{ad hoc}).
  % }
% \end{adjustbox}
\end{table}


\vfill
\printbibliography





\end{document}